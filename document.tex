\documentclass{scrartcl}
\setlength{\parindent}{0pt}
\usepackage{amsthm}
\usepackage{amsmath}
\newtheorem{definitionx}{Definition}
\newenvironment{definition}
{\pushQED{\qed}\renewcommand{\qedsymbol}{$\smiley{}$}\definitionx}
{\popQED\enddefinitionx}

\newtheorem{beispielx}{Beispiel}
\newenvironment{beispiel}
	{\pushQED{\qed}\renewcommand{\qedsymbol}{$\blacksmiley{}$}\beispielx}
	{\popQED\endbeispielx}

\usepackage[utf8]{inputenc}

\usepackage{wasysym}
 
\begin{document}

\section{LMSE Zusammenfassung}
A
\begin{definition}[Prä-Terme]
	Sei $V~=~\{v_1, v_2,...\}$ ein unendliches Alphabet. Die Menge $\Lambda^-$ der \emph{Prä-Terme} ist die Menge aller Strings, die durch die folgende Grammatik erzeugt werden können:\\
	\[\Lambda ^- ::= V \mid (\Lambda ^-~\Lambda ^-) \mid (\lambda V~\Lambda ^-) \]
\end{definition}
\begin{beispiel}[Prä-Terme]
	Folgende Beispiele sind Prä-Terme:
	\begin{enumerate}
		\item $((v_0~v_1)~v_2) \in \Lambda^-$
		\item $(\lambda v_0~(v_0~v_1)) \in \Lambda^-$
		\item $((\lambda v_0~v_0)~v_1) \in \Lambda^-$
		\item $((\lambda v_0~(v_0~v_0))~(\lambda v_1~(v_1~v_1))) \in \Lambda^-$		
	\end{enumerate}
\end{beispiel}
Folgende Terminologien werden verwendet:

\begin{enumerate}
	\item Ein Prä-Term der Form $x$ (ein Element aus $V$) heißt Variable
	\item Ein Prä-Term der Form $(\lambda x~M)$ heißt Abstraktion (über $x$)
	\item Ein Prä-Term der Form $(M~N)$ heißt Anwendung (von $M$ auf $N$) 
\end{enumerate}

Folgende Vereinbarung bezüglich Klammern werden getroffen:

\begin{enumerate}
	\item $(K~L~M)$ für $((K~L)~M)$
	\item $(\lambda x~\lambda y~M)$ für $(\lambda x~(\lambda y~M))$
	\item $(\lambda x~M~N)$ für $(\lambda x~(M~N))$
	\item $(M~\lambda x~N)$ für $(M~(\lambda x~N))$
\end{enumerate}


\begin{beispiel}[Vereinfachte Prä-Terme]
	Vereinfachte Prä-Terme aus Beispiel 1:
	\begin{enumerate}
		\item $v_0~v_1~v_2$
		\item $\lambda v_0.v_0~v_1$
		\item $(\lambda v_0.v_0)~v_1$
		\item $(\lambda v_0.v_0~v_0)~\lambda v_1.v_1~v_1$
	\end{enumerate}
\end{beispiel}

\begin{definition}
	Für eine Menge $M \in \Lambda^-$ ist die Menge $FV(M)$ $\subseteq V$ der \emph{freien Variablen} von $M$ wie folgt definiert:
	\begin{align*} 
		FV(x) &=\{$x$\} \\ 
		FV(\lambda x.P) &= FV(P)\backslash \{x\} \\
		FV(P~Q) &= FV(P) \cup FV(Q)
	\end{align*}
Ist $FM(M)=\emptyset$, dann heißt $M$ \emph{geschlossen}.
\end{definition}

\begin{beispiel}[Freie Variablen] Seien $x,y,z$ Variablen, dann gilt:
	\begin{enumerate}
		\item $FV(x,y,z)~=~\{x,y,z\}$
		\item $FV(\lambda x.x~y)~=~\{y\}$
		\item $FV((\lambda x.x~x)~\lambda y.y~y)~=~\emptyset$
	\end{enumerate}
	
\end{beispiel}

\end{document}

